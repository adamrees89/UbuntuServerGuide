\chapter{Server Naming Scheme}
\label{chp:serverNamingScheme}

Servers are usually named haphazardly by picking a name such as Simpsons characters, or Family Guy characters.  However I came across a guide on the web~\footnote{\url{https://www.mnxsolutions.com/devops/a-proper-server-naming-scheme.html}}.

The basic idea is that there are two types of DNS records for each server, an 'A' record, and a 'CNAME' record.

\section{A Records}

The 'A' record is made by choosing a random name from a unique word list, such as the one in appendix~\ref{app:uniquewordlist}, and using that as a subdomain of your purchased domain.  For example the random word is \textit{vampire}, so the A record would be

\begin{verbatim}
vampire.example.com : 192.168.1.1
\end{verbatim}

\section{CNAME Records}

Now the CNAME record can be used for server administrators to know what each server does at a glance, following the system below in table~\ref{tab:CNAMEsystem}, an using the abbreviations in tables~\ref{tab:evironmentAbb} \&~\ref{tab:purposeAbb}.

\begin{table}[htb]
\centering
\begin{tabular}{|c|c|c|c|}
\hline
\rowcolor{Gray}
Purpose & Environment & Location & Domain\\
\hline
\end{tabular}
\caption{CNAME System}
\label{tab:CNAMEsystem}
\end{table}

So for example our vampire server from earlier would have a record like:

\begin{verbatim}
ssh01.prd.cdf.example.com
\end{verbatim}

So from the beginning, \textit{ssh} states its a SSH server, \textit{prd} states its a production server, and \textit{cdf} states its located in Cardiff, UK.  If you have servers in different countries you can add the two letter country code in front of the three letter location.  I use the UN locode list for location symbols, \href{http://www.unece.org/cefact/locode/service/location.html}{\textbf{UN LOCODE Database}}.

\begin{table}[htb]
\centering
\begin{tabular}{|c|c|}
\hline
\rowcolor{Gray}
Abbreviation & Description\\
\hline
dev & Development\\
tst & Testing\\
stg & Staging\\
prd & Production\\
\hline
\end{tabular}
\caption{Environment Abbreviations}
\label{tab:evironmentAbb}
\end{table}

\begin{table}[htb]
\centering
\begin{tabular}{|c|c|}
\hline
\rowcolor{Gray}
Abbreviation & Description\\
\hline
app & Application Server\\
sql & Database Server\\
ftp & FTP Server\\
mta & Mail Server\\
dns & Name Server\\
cfg & Configuration Management Server\\
mon & Monitoring Server\\
prx & Proxy/Load Balancing\\
ssh & SSH Jump/Bastion\\
sto & Storage Server\\
vcs & Version Control Server\\
vmm & Virtual Machine Manager\\
web & Web Server\\
\hline
\end{tabular}
\caption{Purpose Abbreviations}
\label{tab:purposeAbb}
\end{table}