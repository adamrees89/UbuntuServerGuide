\chapter{Teamspeak Server}
\section{Set-up}

The following procedure is good for installing a teamspeak server. After installation, most of the configuration is done using the teamspeak client software.  Don't forget to copy the access token for use on your client software as admin.

1. Download the teamspeak server file from the website \vspace{0.25cm}

2.Unpack the file (remember to replace the file name if it is different
\begin{lstlisting}
tar xfv teamspeak3-server_linux-x86-3.0.2.tar.gz
\end{lstlisting}

3. Change directory;
\begin{lstlisting}
cd teamspeak3-server_linux-x86
\end{lstlisting}

4. Now run the binary by typing;
\begin{lstlisting}
./ts3server_linux_x86
\end{lstlisting}
and stop it using \keys{\ctrl + c}\\
\\
5. The server commands are as follows;
\begin{lstlisting}
./ts3server_startscript.sh start
./ts3server_startscript.sh stop
./ts3server_startscript.sh status
\end{lstlisting}

Find the internal ip address of the server by typing;\begin{lstlisting}
ip addr
\end{lstlisting}\vspace{0.5cm}

Set up a bash file to start, stop or check status of server
