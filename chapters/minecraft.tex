\chapter{Minecraft Server}
\label{chp:minecraft}

\section{Updating Java}

The next set of commands are run with root privileges, run the following code;

In order to ensure that the java environment is correct, run the command;
\begin{lstlisting}
java -version
\end{lstlisting}

Do the following:

\begin{lstlisting}
update-java-alternatives -l
\end{lstlisting}

This will list off all the various Java VMs that are installed.

\begin{lstlisting}
# update-java-alternatives -l
java-6-openjdk 1061 /usr/lib/jvm/java-6-openjdk
java-6-sun 63 /usr/lib/jvm/java-6-sun
\end{lstlisting}

Next set the proper Java VM to use by typing the following;

\begin{lstlisting}
update-java-alternatives -s java-6-sun
\end{lstlisting}

Now, run 
\begin{lstlisting}
java -version
\end{lstlisting}

It should show the same version information as before.\\
\textit{The rest of these commands are run without root privileges.}

\section{Installing Minecraft Server}

Now download the server file from the minecraft website, and place it into the home directory

To start the server, you will need to use the following command:
\begin{lstlisting}
java -Xmx1024m -Xms1024m -jar minecraft_server.jar nogui
\end{lstlisting}

Once all the text has finished, it will have created a new world file.  Now force the server to save the files and stop.

\begin{lstlisting}
save-all
\end{lstlisting}
which forces the server to save the generated map, then
\begin{lstlisting}
stop
\end{lstlisting}
to shut the server down.

\section{Stopping and Starting the server}

To start the server, first off make sure you are in a screen session by typing 
\begin{lstlisting}
screen -list
\end{lstlisting}
like below:
\begin{lstlisting}
mcserver@mcserver:~$ screen -list
There is a screen on:
 2434.tty1.mcserver        (01/09/2011 12:58:57 PM)        (Attached)
1 Socket in /var/run/screen/S-mcserver.
\end{lstlisting}

This indicates that you have screen session.  If you the console says that no screens are running then type;
\begin{lstlisting}
screen
\end{lstlisting}
to start one.

Once in the screen session, type in the command shown below.
\begin{lstlisting}
java -Xmx1024m -Xms1024m -jar minecraft_server.jar nogui
\end{lstlisting}
To disconnect from the screen session, hit \keys{\ctrl + a} and then the \keys{d} key, this will drop back to the shell prompt where to exit simply type;
\begin{lstlisting}
exit
\end{lstlisting}
to logout. The Minecraft Server will continue to run.

To reconnect with the screen session, type in
\begin{lstlisting}
screen -r
\end{lstlisting}
You will be reconnected to the server and can then perform the following commands:
\begin{lstlisting}
say Server is going down

save-all

stop
\end{lstlisting}
This tells the Minecraft Server to save, shutdown and exit, and tell all the users that it is doing so.

\section{Useful commands in MC Server}

Table~\ref{tab:useful_mc} contains some of the useful commands for maintaining a minecraft server.

\begin{table}[!th]
\centering
\begin{tabular}{cc}
\hline
Command & Description\\
\hline
help  or  ? & shows this message\\
kick \textit{player} & removes a player from the server\\
ban \textit{player} & bans a player from the server\\
pardon \textit{player} & pardons a banned player so that they can connect again\\
ban-ip \textit{ip} & bans an IP address from the server\\
pardon-ip \textit{ip} & pardons a banned IP address so that they can connect again\\
op \textit{player} & turns a player into an op\\
deop \textit{player} & removes op status from a player\\
tp \textit{player1} \textit{player2} & moves one player to the same location as another player\\
give \textit{player} \textit{id} [num] & gives a player a resource\\
tell \textit{player} \textit{message} & sends a private message to a player\\
stop & gracefully stops the server\\
save-all & forces a server-wide level save\\
save-off & disables terrain saving\\
save-on & re-enables terrain saving\\
list & lists all currently connected players\\
say \textit{message} & broadcasts a message to all players\\
\hline
\end{tabular}
\caption{Useful Commands for Minecraft Server}
\label{tab:useful_mc}
\end{table}